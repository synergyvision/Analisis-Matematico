\documentclass[12pt,]{krantz}
\usepackage{lmodern}
\usepackage{amssymb,amsmath}
\usepackage{ifxetex,ifluatex}
\usepackage{fixltx2e} % provides \textsubscript
\ifnum 0\ifxetex 1\fi\ifluatex 1\fi=0 % if pdftex
  \usepackage[T1]{fontenc}
  \usepackage[utf8]{inputenc}
\else % if luatex or xelatex
  \ifxetex
    \usepackage{mathspec}
  \else
    \usepackage{fontspec}
  \fi
  \defaultfontfeatures{Ligatures=TeX,Scale=MatchLowercase}
\fi
% use upquote if available, for straight quotes in verbatim environments
\IfFileExists{upquote.sty}{\usepackage{upquote}}{}
% use microtype if available
\IfFileExists{microtype.sty}{%
\usepackage[]{microtype}
\UseMicrotypeSet[protrusion]{basicmath} % disable protrusion for tt fonts
}{}
\PassOptionsToPackage{hyphens}{url} % url is loaded by hyperref
\usepackage[unicode=true]{hyperref}
\PassOptionsToPackage{usenames,dvipsnames}{color} % color is loaded by hyperref
\hypersetup{
            pdftitle={Introducción al análisis funcional y a la teoría de la medida},
            pdfauthor={Synergy Vision},
            colorlinks=true,
            linkcolor=Maroon,
            citecolor=Blue,
            urlcolor=Blue,
            breaklinks=true}
\urlstyle{same}  % don't use monospace font for urls
\usepackage{natbib}
\bibliographystyle{apalike}
\usepackage{color}
\usepackage{fancyvrb}
\newcommand{\VerbBar}{|}
\newcommand{\VERB}{\Verb[commandchars=\\\{\}]}
\DefineVerbatimEnvironment{Highlighting}{Verbatim}{commandchars=\\\{\}}
% Add ',fontsize=\small' for more characters per line
\usepackage{framed}
\definecolor{shadecolor}{RGB}{248,248,248}
\newenvironment{Shaded}{\begin{snugshade}}{\end{snugshade}}
\newcommand{\KeywordTok}[1]{\textcolor[rgb]{0.13,0.29,0.53}{\textbf{#1}}}
\newcommand{\DataTypeTok}[1]{\textcolor[rgb]{0.13,0.29,0.53}{#1}}
\newcommand{\DecValTok}[1]{\textcolor[rgb]{0.00,0.00,0.81}{#1}}
\newcommand{\BaseNTok}[1]{\textcolor[rgb]{0.00,0.00,0.81}{#1}}
\newcommand{\FloatTok}[1]{\textcolor[rgb]{0.00,0.00,0.81}{#1}}
\newcommand{\ConstantTok}[1]{\textcolor[rgb]{0.00,0.00,0.00}{#1}}
\newcommand{\CharTok}[1]{\textcolor[rgb]{0.31,0.60,0.02}{#1}}
\newcommand{\SpecialCharTok}[1]{\textcolor[rgb]{0.00,0.00,0.00}{#1}}
\newcommand{\StringTok}[1]{\textcolor[rgb]{0.31,0.60,0.02}{#1}}
\newcommand{\VerbatimStringTok}[1]{\textcolor[rgb]{0.31,0.60,0.02}{#1}}
\newcommand{\SpecialStringTok}[1]{\textcolor[rgb]{0.31,0.60,0.02}{#1}}
\newcommand{\ImportTok}[1]{#1}
\newcommand{\CommentTok}[1]{\textcolor[rgb]{0.56,0.35,0.01}{\textit{#1}}}
\newcommand{\DocumentationTok}[1]{\textcolor[rgb]{0.56,0.35,0.01}{\textbf{\textit{#1}}}}
\newcommand{\AnnotationTok}[1]{\textcolor[rgb]{0.56,0.35,0.01}{\textbf{\textit{#1}}}}
\newcommand{\CommentVarTok}[1]{\textcolor[rgb]{0.56,0.35,0.01}{\textbf{\textit{#1}}}}
\newcommand{\OtherTok}[1]{\textcolor[rgb]{0.56,0.35,0.01}{#1}}
\newcommand{\FunctionTok}[1]{\textcolor[rgb]{0.00,0.00,0.00}{#1}}
\newcommand{\VariableTok}[1]{\textcolor[rgb]{0.00,0.00,0.00}{#1}}
\newcommand{\ControlFlowTok}[1]{\textcolor[rgb]{0.13,0.29,0.53}{\textbf{#1}}}
\newcommand{\OperatorTok}[1]{\textcolor[rgb]{0.81,0.36,0.00}{\textbf{#1}}}
\newcommand{\BuiltInTok}[1]{#1}
\newcommand{\ExtensionTok}[1]{#1}
\newcommand{\PreprocessorTok}[1]{\textcolor[rgb]{0.56,0.35,0.01}{\textit{#1}}}
\newcommand{\AttributeTok}[1]{\textcolor[rgb]{0.77,0.63,0.00}{#1}}
\newcommand{\RegionMarkerTok}[1]{#1}
\newcommand{\InformationTok}[1]{\textcolor[rgb]{0.56,0.35,0.01}{\textbf{\textit{#1}}}}
\newcommand{\WarningTok}[1]{\textcolor[rgb]{0.56,0.35,0.01}{\textbf{\textit{#1}}}}
\newcommand{\AlertTok}[1]{\textcolor[rgb]{0.94,0.16,0.16}{#1}}
\newcommand{\ErrorTok}[1]{\textcolor[rgb]{0.64,0.00,0.00}{\textbf{#1}}}
\newcommand{\NormalTok}[1]{#1}
\usepackage{longtable,booktabs}
% Fix footnotes in tables (requires footnote package)
\IfFileExists{footnote.sty}{\usepackage{footnote}\makesavenoteenv{long table}}{}
\usepackage{graphicx,grffile}
\makeatletter
\def\maxwidth{\ifdim\Gin@nat@width>\linewidth\linewidth\else\Gin@nat@width\fi}
\def\maxheight{\ifdim\Gin@nat@height>\textheight\textheight\else\Gin@nat@height\fi}
\makeatother
% Scale images if necessary, so that they will not overflow the page
% margins by default, and it is still possible to overwrite the defaults
% using explicit options in \includegraphics[width, height, ...]{}
\setkeys{Gin}{width=\maxwidth,height=\maxheight,keepaspectratio}
\IfFileExists{parskip.sty}{%
\usepackage{parskip}
}{% else
\setlength{\parindent}{0pt}
\setlength{\parskip}{6pt plus 2pt minus 1pt}
}
\setlength{\emergencystretch}{3em}  % prevent overfull lines
\providecommand{\tightlist}{%
  \setlength{\itemsep}{0pt}\setlength{\parskip}{0pt}}
\setcounter{secnumdepth}{5}
% Redefines (sub)paragraphs to behave more like sections
\ifx\paragraph\undefined\else
\let\oldparagraph\paragraph
\renewcommand{\paragraph}[1]{\oldparagraph{#1}\mbox{}}
\fi
\ifx\subparagraph\undefined\else
\let\oldsubparagraph\subparagraph
\renewcommand{\subparagraph}[1]{\oldsubparagraph{#1}\mbox{}}
\fi

% set default figure placement to htbp
\makeatletter
\def\fps@figure{htbp}
\makeatother

\usepackage[T1]{fontenc}
\usepackage[utf8]{inputenc} % recommended encoding
\usepackage[spanish]{babel}
\usepackage{booktabs}
\usepackage{longtable}
\usepackage[bf,singlelinecheck=off]{caption}

%\setmainfont[UprightFeatures={SmallCapsFont=AlegreyaSC-Regular}]{Alegreya}

\usepackage{framed,color}
\definecolor{shadecolor}{RGB}{248,248,248}

\renewcommand{\textfraction}{0.05}
\renewcommand{\topfraction}{0.8}
\renewcommand{\bottomfraction}{0.8}
\renewcommand{\floatpagefraction}{0.75}

\renewenvironment{quote}{\begin{VF}}{\end{VF}}
\let\oldhref\href
\renewcommand{\href}[2]{#2\footnote{\url{#1}}}

\ifxetex
  \usepackage{letltxmacro}
  \setlength{\XeTeXLinkMargin}{1pt}
  \LetLtxMacro\SavedIncludeGraphics\includegraphics
  \def\includegraphics#1#{% #1 catches optional stuff (star/opt. arg.)
    \IncludeGraphicsAux{#1}%
  }%
  \newcommand*{\IncludeGraphicsAux}[2]{%
    \XeTeXLinkBox{%
      \SavedIncludeGraphics#1{#2}%
    }%
  }%
\fi

\makeatletter
\newenvironment{kframe}{%
\medskip{}
\setlength{\fboxsep}{.8em}
 \def\at@end@of@kframe{}%
 \ifinner\ifhmode%
  \def\at@end@of@kframe{\end{minipage}}%
  \begin{minipage}{\columnwidth}%
 \fi\fi%
 \def\FrameCommand##1{\hskip\@totalleftmargin \hskip-\fboxsep
 \colorbox{shadecolor}{##1}\hskip-\fboxsep
     % There is no \\@totalrightmargin, so:
     \hskip-\linewidth \hskip-\@totalleftmargin \hskip\columnwidth}%
 \MakeFramed {\advance\hsize-\width
   \@totalleftmargin\z@ \linewidth\hsize
   \@setminipage}}%
 {\par\unskip\endMakeFramed%
 \at@end@of@kframe}
\makeatother

\renewenvironment{Shaded}{\begin{kframe}}{\end{kframe}}

\newenvironment{rmdblock}[1]
  {
  \begin{itemize}
  \renewcommand{\labelitemi}{
    \raisebox{-.7\height}[0pt][0pt]{
      {\setkeys{Gin}{width=3em,keepaspectratio}\includegraphics{images/#1}}
    }
  }
  \setlength{\fboxsep}{1em}
  \begin{kframe}
  \item
  }
  {
  \end{kframe}
  \end{itemize}
  }
\newenvironment{rmdnote}
  {\begin{rmdblock}{note}}
  {\end{rmdblock}}
\newenvironment{rmdcaution}
  {\begin{rmdblock}{caution}}
  {\end{rmdblock}}
\newenvironment{rmdimportant}
  {\begin{rmdblock}{important}}
  {\end{rmdblock}}
\newenvironment{rmdtip}
  {\begin{rmdblock}{tip}}
  {\end{rmdblock}}
\newenvironment{rmdwarning}
  {\begin{rmdblock}{warning}}
  {\end{rmdblock}}

\usepackage{makeidx}
\makeindex

\urlstyle{tt}

\usepackage{amsthm}
\makeatletter
\def\thm@space@setup{%
  \thm@preskip=8pt plus 2pt minus 4pt
  \thm@postskip=\thm@preskip
}
\makeatother

\frontmatter

\title{Introducción al análisis funcional y a la teoría de la medida}
\providecommand{\subtitle}[1]{}
\subtitle{Ciencia de los Datos Financieros}
\author{Synergy Vision}
\date{2018-05-03}

\usepackage{amsthm}
\newtheorem{theorem}{Teorema}[chapter]
\newtheorem{lemma}{Lema}[chapter]
\theoremstyle{definition}
\newtheorem{definition}{Definición}[chapter]
\newtheorem{corollary}{Corolario}[chapter]
\newtheorem{proposition}{Proposición}[chapter]
\theoremstyle{definition}
\newtheorem{example}{Ejemplo}[chapter]
\theoremstyle{definition}
\newtheorem{exercise}{Ejercicio}[chapter]
\theoremstyle{remark}
\newtheorem*{remark}{Nota}
\newtheorem*{solution}{Solución}
\let\BeginKnitrBlock\begin \let\EndKnitrBlock\end
\begin{document}
\maketitle

%\cleardoublepage\newpage\thispagestyle{empty}\null
%\cleardoublepage\newpage\thispagestyle{empty}\null
%\cleardoublepage\newpage
\thispagestyle{empty}


\setlength{\abovedisplayskip}{-5pt}
\setlength{\abovedisplayshortskip}{-5pt}

{
\hypersetup{linkcolor=black}
\setcounter{tocdepth}{2}
\tableofcontents
}
\listoftables
\listoffigures
\chapter*{Prefacio}\label{prefacio}


\includegraphics{images/by-nc-sa.png}\\
La versión en línea de este libro se comparte bajo la licencia
\href{http://creativecommons.org/licenses/by-nc-sa/4.0/}{Creative
Commons Attribution-NonCommercial-ShareAlike 4.0 International License}.

\section*{¿Por qué leer este libro?}\label{por-que-leer-este-libro}


Este libro es el resultado de enfocarnos en proveer la mayor cantidad de
material sobre Análisis y teoría de la medida con un desarrollo teórico
lo más explícito posible, con el valor agregado de incorporar ejemplos
de las finanzas y la programación en \texttt{R}. Finalmente tenemos un
libro interactivo que ofrece una experiencia de aprendizaje distinta e
innovadora.

El un mundo abierto, ya no es tanto el acceso a la información, sino el
acceso al conocimiento.

Es mucha la literatura, pero son pocas las opciones donde se pueda
navegar el libro de forma amigable y además contar con ejemplos en
\texttt{R} y ejercicios interactivos, además del contenido multimedia.
Esperamos que ésta sea un contribución sobre nuevas prácticas para
publicar el contenido y darle vida, crear una experiencia distinta, una
experiencia interactiva y visual. El reto es darle vida al contenido
asistidos con las herramientas de Internet.

Finalmente este es un intento de ofrecer otra visión sobre la enseñanza
y la generación de material más accesible. Estamos en un mundo
multidisciplinado, es por ello que ahora hay que generar contenido que
conjugue en un mismo lugar las matemáticas, estadística, finanzas y la
computación.

Lo dejamos público ya que las herramientas que usamos para ensamblarlo
son abiertas y públicas.

\section*{Estructura del libro}\label{estructura-del-libro}


TODO: Describir la estructura

\section*{Información sobre los programas y
convenciones}\label{informacion-sobre-los-programas-y-convenciones}
\addcontentsline{toc}{section}{Información sobre los programas y
convenciones}

Este libro es posible gracias a una gran cantidad de desarrolladores que
contribuyen en la construcción de herramientas para generar documentos
enriquecidos e interactivos. En particular al autor de los paquetes
Yihui Xie xie2015.

\section*{Prácticas interactivas con
R}\label{practicas-interactivas-con-r}


Vamos a utilizar el paquete
\href{https://github.com/datacamp/tutorial}{Datacamp Tutorial} que
utiliza la librería en JavaScript
\href{https://github.com/datacamp/datacamp-light}{Datacamp Light} para
crear ejercicios y prácticas con \texttt{R}. De esta forma el libro es
completamente interactivo y con prácticas incluidas. De esta forma
estamos creando una experiencia única de aprendizaje en línea.

eyJsYW5ndWFnZSI6InIiLCJwcmVfZXhlcmNpc2VfY29kZSI6ImIgPC0gNSIsInNhbXBsZSI6IiMgQ3JlYSB1bmEgdmFyaWFibGUgYSwgaWd1YWwgYSA1XG5cblxuIyBNdWVzdHJhIGVsIHZhbG9yIGRlIGEiLCJzb2x1dGlvbiI6IiMgQ3JlYSB1bmEgdmFyaWFibGUgYSwgaWd1YWwgYSA1XG5hIDwtIDVcblxuIyBNdWVzdHJhIGVsIHZhbG9yIGRlIGFcbmEiLCJzY3QiOiJ0ZXN0X29iamVjdChcImFcIilcbnRlc3Rfb3V0cHV0X2NvbnRhaW5zKFwiYVwiLCBpbmNvcnJlY3RfbXNnID0gXCJBc2VnJnVhY3V0ZTtyYXRlIGRlIG1vc3RyYXIgZWwgdmFsb3IgZGUgYGFgLlwiKVxuc3VjY2Vzc19tc2coXCJFeGNlbGVudGUhXCIpIn0=

\section*{Agradecimientos}\label{agradecimientos}


A todo el equipo de Synergy Vision que no deja de soñar. Hay que hacer
lo que pocos hacen, insistir, insistir hasta alcanzar. Lo más importante
es concretar las ideas. La idea es sólo el inicio y solo vale cuando se
concreta.

\BeginKnitrBlock{flushright}
Synergy Vision, Caracas, Venezuela
\EndKnitrBlock{flushright}

\chapter*{Acerca del Autor}\label{acerca-del-autor}


Este material es un esfuerzo de equipo en Synergy Vision,
(\url{http://synergy.vision/nosotros/}).

El propósito de este material es ofrecer una experiencia de aprendizaje
distinta y enfocada en el estudiante. El propósito es que realmente
aprenda y practique con mucha intensidad. La idea es cambiar el modelo
de clases magistrales y ofrecer una experiencia más centrada en el
estudiante y menos centrado en el profesor. Para los temas más técnicos
y avanzados es necesario trabajar de la mano con el estudiante y
asistirlo en el proceso de aprendizaje con prácticas guiadas, material
en línea e interactivo, videos, evaluación contínua de brechas y
entendimiento, entre otros, para procurar el dominio de la materia.

Nuestro foco es la Ciencia de los Datos Financieros y para ello se
desarrollará material sobre: \textbf{Probabilidad y Estadística
Matemática en R}, \textbf{Programación Científica en R},
\textbf{Mercados}, \textbf{Inversiones y Trading}, \textbf{Datos y
Modelos Financieros en R}, \textbf{Renta Fija}, \textbf{Inmunización de
Carteras de Renta Fija}, \textbf{Teoría de Riesgo en R},
\textbf{Finanzas Cuantitativas}, \textbf{Ingeniería Financiera},
\textbf{Procesos Estocásticos en R}, \textbf{Series de Tiempo en R},
\textbf{Ciencia de los Datos}, \textbf{Ciencia de los Datos
Financieros}, \textbf{Simulación en R}, \textbf{Desarrollo de
Aplicaciones Interactivas en R}, \textbf{Minería de Datos},
\textbf{Aprendizaje Estadístico}, \textbf{Estadística Multivariante},
\textbf{Riesgo de Crédito}, \textbf{Riesgo de Liquidez}, \textbf{Riesgo
de Mercado}, \textbf{Riesgo Operacional}, \textbf{Riesgo de Cambio},
\textbf{Análisis Técnico}, \textbf{Inversión Visual}, \textbf{Finanzas},
\textbf{Finanzas Corporativas}, \textbf{Valoración}, \textbf{Teoría de
Portafolio}, entre otros.

Nuestra cuenta de Twitter es (\url{https://twitter.com/bysynergyvision})
y nuestros repositorios están en GitHub
(\url{https://github.com/synergyvision}).

\textbf{Somos Científicos de Datos Financieros}

\mainmatter

\chapter{Introducción}\label{introduccion}

\chapter{Conjuntos}\label{conjuntos}

\section{Operaciones}\label{operaciones}

\section{Relaciones}\label{relaciones}

\section{Funciones}\label{funciones}

\section{Cardinalidad}\label{cardinalidad}

\chapter{Sistema numérico real y
complejo}\label{sistema-numerico-real-y-complejo}

\section{Introducción}\label{introduccion-1}

\section{\texorpdfstring{Propiedades algebraicas de
\(\mathbb{R}\)}{Propiedades algebraicas de \textbackslash{}mathbb\{R\}}}\label{propiedades-algebraicas-de-mathbbr}

\section{\texorpdfstring{Estructura de orden de
\(\mathbb{R}\)}{Estructura de orden de \textbackslash{}mathbb\{R\}}}\label{estructura-de-orden-de-mathbbr}

\section{\texorpdfstring{Propiedades de completitud de
\(\mathbb{R}\)}{Propiedades de completitud de \textbackslash{}mathbb\{R\}}}\label{propiedades-de-completitud-de-mathbbr}

\section{Inducción matemática}\label{induccion-matematica}

\section{Espacios euclídeos}\label{espacios-euclideos}

\chapter{Estructuras algebraicas}\label{estructuras-algebraicas}

\section{Semigrupos y grupos}\label{semigrupos-y-grupos}

\section{Espacios vectoriales}\label{espacios-vectoriales}

\section{Transformaciones lineales}\label{transformaciones-lineales}

\section{Espacios vectoriales
cocientes}\label{espacios-vectoriales-cocientes}

\section{Álgebras}\label{algebras}

\chapter{Sucesiones numéricas}\label{sucesiones-numericas}

\section{Límite de una sucesión}\label{limite-de-una-sucesion}

\section{Sucesiones monótonas}\label{sucesiones-monotonas}

\section{Subsucesiones y sucesiones de
Cauchy}\label{subsucesiones-y-sucesiones-de-cauchy}

\section{Límites inferior y superior}\label{limites-inferior-y-superior}

\chapter{Sucesiones y series}\label{sucesiones-y-series}

\section{Límite de una función}\label{limite-de-una-funcion}

\section{Límites inferior y
superior}\label{limites-inferior-y-superior-1}

\section{Funciones contínuas}\label{funciones-continuas}

\section{Propiedades de las funciones
contínuas}\label{propiedades-de-las-funciones-continuas}

\section{Continuidad uniforme}\label{continuidad-uniforme}

\chapter{Diferenciación}\label{diferenciacion}

\section{Definición y ejemplos}\label{definicion-y-ejemplos}

\section{El teorema del valor medio}\label{el-teorema-del-valor-medio}

\section{Funciones convexas}\label{funciones-convexas}

\section{Funciones inversas}\label{funciones-inversas}

\section{Regla de L'Hospital}\label{regla-de-lhospital}

\section{\texorpdfstring{Teorema de Taylor en
\(\mathbb{R}\)}{Teorema de Taylor en \textbackslash{}mathbb\{R\}}}\label{teorema-de-taylor-en-mathbbr}

\section{Método de Newton}\label{metodo-de-newton}

\chapter{Integración de Riemann}\label{integracion-de-riemann}

\section{Integral de Riemann-Darboux}\label{integral-de-riemann-darboux}

\section{Propiedades de la integral}\label{propiedades-de-la-integral}

\section{Evaluación de la integral}\label{evaluacion-de-la-integral}

\section{Fórmula de Stirling}\label{formula-de-stirling}

\section{Teoremas del valor medio, versión
integral}\label{teoremas-del-valor-medio-version-integral}

\section{Estimación de la integral}\label{estimacion-de-la-integral}

\section{Integrales impropias}\label{integrales-impropias}

\section{La integrabilidad según
Riemann}\label{la-integrabilidad-segun-riemann}

\section{Funciones a variación
acotada}\label{funciones-a-variacion-acotada}

\section{La integral de
Riemann-Stieltjes}\label{la-integral-de-riemann-stieltjes}

\chapter{Series numéricas infinitas}\label{series-numericas-infinitas}

\section{Definición y ejemplos}\label{definicion-y-ejemplos-1}

\section{Series con términos
no-negativos}\label{series-con-terminos-no-negativos}

\section{Criterios de convergencia}\label{criterios-de-convergencia}

\section{Convergencia condicional y
absoluta}\label{convergencia-condicional-y-absoluta}

\section{Sucesiones dobles y series}\label{sucesiones-dobles-y-series}

\chapter{Sucesiones y series de
funciones}\label{sucesiones-y-series-de-funciones}

\section{Convergencia de sucesiones de
funciones}\label{convergencia-de-sucesiones-de-funciones}

\section{Propiedades del límite de
funciones}\label{propiedades-del-limite-de-funciones}

\section{Convergencia de las series de
funciones}\label{convergencia-de-las-series-de-funciones}

\section{Series de potencias}\label{series-de-potencias}

\chapter{Funciones en varias
variables}\label{funciones-en-varias-variables}

\section{Transformaciones lineales}\label{transformaciones-lineales-1}

\section{Diferenciación}\label{diferenciacion-1}

\section{El principio de la
contracción}\label{el-principio-de-la-contraccion}

\section{El teorema de la función
inversa}\label{el-teorema-de-la-funcion-inversa}

\section{El teorema de la función
implícita}\label{el-teorema-de-la-funcion-implicita}

\section{Teorema del rango}\label{teorema-del-rango}

\section{Determinantes}\label{determinantes}

\section{Derivadas de orden superior}\label{derivadas-de-orden-superior}

\section{Diferenciación de
integrales}\label{diferenciacion-de-integrales}

\chapter{Integración de formas
diferenciales}\label{integracion-de-formas-diferenciales}

\section{Integración}\label{integracion}

\section{Aplicaciones primitivas}\label{aplicaciones-primitivas}

\section{Cambio de variables}\label{cambio-de-variables}

\section{Formas diferenciales}\label{formas-diferenciales}

\section{Cadenas y símplices}\label{cadenas-y-simplices}

\section{Teorema de Stoke}\label{teorema-de-stoke}

\section{Formas cerradas y formas
exactas}\label{formas-cerradas-y-formas-exactas}

\section{Análisis vectorial}\label{analisis-vectorial}

\chapter{Funciones especiales}\label{funciones-especiales}

\section{Series de potencia}\label{series-de-potencia}

\section{Funciones exponenciales y
logarítmicas}\label{funciones-exponenciales-y-logaritmicas}

\section{Funciones trigonométricas}\label{funciones-trigonometricas}

\section{Completitud algebraica del cuerpo de los
complejos}\label{completitud-algebraica-del-cuerpo-de-los-complejos}

\section{Series de Fourier}\label{series-de-fourier}

\chapter{Espacios lineales normados}\label{espacios-lineales-normados}

\section{Normas y seminormas}\label{normas-y-seminormas}

\section{Completación de un espacio
normado}\label{completacion-de-un-espacio-normado}

\section{Series infinitas en espacios
normados}\label{series-infinitas-en-espacios-normados}

\section{Sumas no ordenadas en espacios
normados}\label{sumas-no-ordenadas-en-espacios-normados}

\section{Trnasformaciones lineales
acotadas}\label{trnasformaciones-lineales-acotadas}

\section{Álgebras de Banach}\label{algebras-de-banach}

\chapter{Espacios topológicos}\label{espacios-topologicos}

\section{Abiertos y cerrados}\label{abiertos-y-cerrados}

\section{Sistemas de entornos}\label{sistemas-de-entornos}

\section{Bases de entornos}\label{bases-de-entornos}

\section{Topología relativa}\label{topologia-relativa}

\section{Nets}\label{nets}

\chapter{Continuidad en espacios
topológicos}\label{continuidad-en-espacios-topologicos}

\section{Propiedades generales}\label{propiedades-generales}

\section{Topologías iniciales}\label{topologias-iniciales}

\section{Topología producto}\label{topologia-producto}

\section{Topología cociente}\label{topologia-cociente}

\section{Espacio de funciones
contínuas}\label{espacio-de-funciones-continuas}

\section{Conjuntos F-sigma y G-delta}\label{conjuntos-f-sigma-y-g-delta}

\chapter{Espacios topológicos
normados}\label{espacios-topologicos-normados}

\section{Lema de Urysohn}\label{lema-de-urysohn}

\section{Teorema de extensión de
Tietze}\label{teorema-de-extension-de-tietze}

\chapter{Espacios topológicos
compactos}\label{espacios-topologicos-compactos}

\section{Convergencia en espacios
compactos}\label{convergencia-en-espacios-compactos}

\section{Compacidad del producto
cartesiano}\label{compacidad-del-producto-cartesiano}

\section{Continuidad y compacidad}\label{continuidad-y-compacidad}

\chapter{Espacios métricos totalmente
acotados}\label{espacios-metricos-totalmente-acotados}

\chapter{Equicontinuidad}\label{equicontinuidad}

\chapter{El teorema de
Stone-Weierstrass}\label{el-teorema-de-stone-weierstrass}

\chapter{Espacios toplógicos localmente
compactos}\label{espacios-toplogicos-localmente-compactos}

\section{Propiedades generales}\label{propiedades-generales-1}

\section{Funciones a soporte
compacto}\label{funciones-a-soporte-compacto}

\section{Funciones que se anulan al
infinito}\label{funciones-que-se-anulan-al-infinito}

\section{Compactificación a un punto}\label{compactificacion-a-un-punto}

\chapter{Espacios de Hilbert}\label{espacios-de-hilbert}

\section{Definición y ejemplos}\label{definicion-y-ejemplos-2}

\section{Ortogonalidad}\label{ortogonalidad}

\section{Separación de conjuntos
convexos}\label{separacion-de-conjuntos-convexos}

\section{Bases ortonormadas}\label{bases-ortonormadas}

\section{Convergencia débil}\label{convergencia-debil}

\section{Operadores contínuos y
compactos}\label{operadores-continuos-y-compactos}

\section{Teorema espectral de
Hilbert}\label{teorema-espectral-de-hilbert}

\chapter{Espacio de funciones
diferenciables}\label{espacio-de-funciones-diferenciables}

\chapter{Particiones de la unidad}\label{particiones-de-la-unidad}

\chapter{Conexidad}\label{conexidad}

\chapter{Espacios de Banach}\label{espacios-de-banach}

\section{Espacios normados}\label{espacios-normados}

\section{Separación de conjuntos
convexos}\label{separacion-de-conjuntos-convexos-1}

\section{Teorema de prolongamiento}\label{teorema-de-prolongamiento}

\section{\texorpdfstring{Duales de los espacios
\(\ell^p\)}{Duales de los espacios \textbackslash{}ell\^{}p}}\label{duales-de-los-espacios-ellp}

\section{Convergencia débil}\label{convergencia-debil-1}

\section{Teorema de Banach-Steinhaus}\label{teorema-de-banach-steinhaus}

\section{Espacios reflexivos}\label{espacios-reflexivos}

\section{Operadores contínuos y
compactos}\label{operadores-continuos-y-compactos-1}

\section{Teorema de Fredholm-Riesz}\label{teorema-de-fredholm-riesz}

\section{Aplicaciones abiertos y grafos
cerrados}\label{aplicaciones-abiertos-y-grafos-cerrados}

\section{Caso complejo}\label{caso-complejo}

\chapter{Espacios convexos}\label{espacios-convexos}

\section{Familias de seminormas}\label{familias-de-seminormas}

\section{Teorema de separación y de
prolongamiento}\label{teorema-de-separacion-y-de-prolongamiento}

\section{Teorema de Krein-Milman}\label{teorema-de-krein-milman}

\chapter{Conjuntos medibles}\label{conjuntos-medibles}

\section{Introducción}\label{introduccion-2}

\section{Espacios medibles}\label{espacios-medibles}

\section{Medidas}\label{medidas}

\section{Espacios medibles completos}\label{espacios-medibles-completos}

\section{Medida externa y
medibilidad}\label{medida-externa-y-medibilidad}

\section{Extensión de una medida}\label{extension-de-una-medida}

\section{Medida de Lebesgue}\label{medida-de-lebesgue}

\section{Medida de Lebesgue
Stieltjes}\label{medida-de-lebesgue-stieltjes}

\section{Conjuntos especiales}\label{conjuntos-especiales}

\chapter{Funciones medibles}\label{funciones-medibles}

\section{Transformaciones medibles}\label{transformaciones-medibles}

\section{Funciones numéricas
medibles}\label{funciones-numericas-medibles}

\section{Funciones simples}\label{funciones-simples}

\section{Convergencia de funciones
medibles}\label{convergencia-de-funciones-medibles}

\chapter{Integración}\label{integracion-1}

\section{Construcción de la integral}\label{construccion-de-la-integral}

\section{Propiedades básicas de la
integral}\label{propiedades-basicas-de-la-integral}

\section{\texorpdfstring{Conexión con la integral de Riemann en
\(\mathbb{R}^n\)}{Conexión con la integral de Riemann en \textbackslash{}mathbb\{R\}\^{}n}}\label{conexion-con-la-integral-de-riemann-en-mathbbrn}

\section{Teoremas de convergencia}\label{teoremas-de-convergencia}

\section{Integración sobre una medida
producto}\label{integracion-sobre-una-medida-producto}

\section{Aplicaciones del teorema de
Fubini}\label{aplicaciones-del-teorema-de-fubini}

\chapter{\texorpdfstring{Espacios
\(L^p\)}{Espacios L\^{}p}}\label{espacios-lp}

\section{Definición y propiedades
generales}\label{definicion-y-propiedades-generales}

\section{\texorpdfstring{Aproximación en
\(L^p\)}{Aproximación en L\^{}p}}\label{aproximacion-en-lp}

\section{\texorpdfstring{Convergencia en
\(L^p\)}{Convergencia en L\^{}p}}\label{convergencia-en-lp}

\section{Integrabilidad uniforme}\label{integrabilidad-uniforme}

\section{Funciones convexas y desigualdad de
Jensen}\label{funciones-convexas-y-desigualdad-de-jensen}

\chapter{Diferenciación}\label{diferenciacion-2}

\section{Medidas con signo}\label{medidas-con-signo}

\section{Medidas complejas}\label{medidas-complejas}

\section{Continuidad absoluta de
medidas}\label{continuidad-absoluta-de-medidas}

\section{Diferenciación de medidas}\label{diferenciacion-de-medidas}

\section{Funciones a variación
acotada}\label{funciones-a-variacion-acotada-1}

\section{Funciones absolutamente
contínuas}\label{funciones-absolutamente-continuas}

\chapter{\texorpdfstring{Análisis de Fourier en
\(\mathbb{R}^n\)}{Análisis de Fourier en \textbackslash{}mathbb\{R\}\^{}n}}\label{analisis-de-fourier-en-mathbbrn}

\section{Convolución de funciones}\label{convolucion-de-funciones}

\section{La transformada de Fourier}\label{la-transformada-de-fourier}

\section{Funciones de rápido
decrecimiento}\label{funciones-de-rapido-decrecimiento}

\section{\texorpdfstring{Análisis de Fourier de medidas en
\(\mathbb{R}^n\)}{Análisis de Fourier de medidas en \textbackslash{}mathbb\{R\}\^{}n}}\label{analisis-de-fourier-de-medidas-en-mathbbrn}

\chapter{Medidas en espacios localmente
compactos}\label{medidas-en-espacios-localmente-compactos}

\section{Medidas de Radon}\label{medidas-de-radon}

\section{El teorema de representación de
Riesz}\label{el-teorema-de-representacion-de-riesz}

\section{Productos de medidas de
Radon}\label{productos-de-medidas-de-radon}

\section{El operador dual}\label{el-operador-dual}

\section{Operadores compactos}\label{operadores-compactos}

\chapter{Espacios localmente
convexos}\label{espacios-localmente-convexos}

\section{Propiedades generales}\label{propiedades-generales-2}

\section{Funcionales lineales
contínuos}\label{funcionales-lineales-continuos}

\section{Teoremas de separación de
Hahn-Banach}\label{teoremas-de-separacion-de-hahn-banach}

\section{Algunas construcciones}\label{algunas-construcciones}

\chapter{Topologías débiles en espacios
normados}\label{topologias-debiles-en-espacios-normados}

\section{Topología débil}\label{topologia-debil}

\section{\texorpdfstring{Topología
débil\(^*\)}{Topología débil\^{}*}}\label{topologia-debil-1}

\section{Espacios reflexivos}\label{espacios-reflexivos-1}

\section{Espacios uniformemente
convexos}\label{espacios-uniformemente-convexos}

\chapter{Espacios de Hilbert}\label{espacios-de-hilbert-1}

\section{Principios generales}\label{principios-generales}

\section{Ortogonalidad}\label{ortogonalidad-1}

\section{Bases ortonormales}\label{bases-ortonormales}

\section{El adjunto del espacio de
Hilbert}\label{el-adjunto-del-espacio-de-hilbert}

\chapter{Teoría de operadores}\label{teoria-de-operadores}

\section{Tipos de operadores}\label{tipos-de-operadores}

\section{Operadores compactos y de rango
finito}\label{operadores-compactos-y-de-rango-finito}

\section{El teorema espectral para operadores normales
compactos}\label{el-teorema-espectral-para-operadores-normales-compactos}

\section{\texorpdfstring{El álgebra del grupo
\(L^1\)}{El álgebra del grupo L\^{}1}}\label{el-algebra-del-grupo-l1}

\section{Representaciones}\label{representaciones}

\section{Grupos abelianos localmente
compactos}\label{grupos-abelianos-localmente-compactos}

\chapter{Análisis en semigrupos}\label{analisis-en-semigrupos}

\section{Semigrupos con topologías}\label{semigrupos-con-topologias}

\section{Funciones debilmente casi
periódicas}\label{funciones-debilmente-casi-periodicas}

\section{La estructura de los semigrupos
compactos}\label{la-estructura-de-los-semigrupos-compactos}

\section{Funciones fuertemente casi
periódicas}\label{funciones-fuertemente-casi-periodicas}

\section{Semigrupo de operadores}\label{semigrupo-de-operadores}

\chapter{Teoría de probabilidades}\label{teoria-de-probabilidades}

\section{Variables aleatorias}\label{variables-aleatorias}

\section{Independencia}\label{independencia}

\section{Esperanza condicional}\label{esperanza-condicional}

\section{Sucesiones de variables aleatorias
independientes}\label{sucesiones-de-variables-aleatorias-independientes}

\section{Martingalas a tiempo
discreto}\label{martingalas-a-tiempo-discreto}

\section{Procesos estocásticos}\label{procesos-estocasticos}

\section{Movimiento browniano}\label{movimiento-browniano}

\section{Integración estocástica}\label{integracion-estocastica}

\section{Aplicación a las finanzas}\label{aplicacion-a-las-finanzas}

\chapter{Apéndice}\label{apendice}

\cleardoublepage 

\appendix \addcontentsline{toc}{chapter}{\appendixname}


\chapter{Software Tools}\label{software-tools}

For those who are not familiar with software packages required for using
R Markdown, we give a brief introduction to the installation and
maintenance of these packages.

\section{R and R packages}\label{r-and-r-packages}

R can be downloaded and installed from any CRAN (the Comprehensive R
Archive Network) mirrors, e.g., \url{https://cran.rstudio.com}. Please
note that there will be a few new releases of R every year, and you may
want to upgrade R occasionally.

To install the \textbf{bookdown} package, you can type this in R:

\begin{Shaded}
\begin{Highlighting}[]
\KeywordTok{install.packages}\NormalTok{(}\StringTok{"bookdown"}\NormalTok{)}
\end{Highlighting}
\end{Shaded}

This installs all required R packages. You can also choose to install
all optional packages as well, if you do not care too much about whether
these packages will actually be used to compile your book (such as
\textbf{htmlwidgets}):

\begin{Shaded}
\begin{Highlighting}[]
\KeywordTok{install.packages}\NormalTok{(}\StringTok{"bookdown"}\NormalTok{, }\DataTypeTok{dependencies =} \OtherTok{TRUE}\NormalTok{)}
\end{Highlighting}
\end{Shaded}

If you want to test the development version of \textbf{bookdown} on
GitHub, you need to install \textbf{devtools} first:

\begin{Shaded}
\begin{Highlighting}[]
\ControlFlowTok{if}\NormalTok{ (}\OperatorTok{!}\KeywordTok{requireNamespace}\NormalTok{(}\StringTok{'devtools'}\NormalTok{)) }\KeywordTok{install.packages}\NormalTok{(}\StringTok{'devtools'}\NormalTok{)}
\NormalTok{devtools}\OperatorTok{::}\KeywordTok{install_github}\NormalTok{(}\StringTok{'rstudio/bookdown'}\NormalTok{)}
\end{Highlighting}
\end{Shaded}

R packages are also often constantly updated on CRAN or GitHub, so you
may want to update them once in a while:

\begin{Shaded}
\begin{Highlighting}[]
\KeywordTok{update.packages}\NormalTok{(}\DataTypeTok{ask =} \OtherTok{FALSE}\NormalTok{)}
\end{Highlighting}
\end{Shaded}

Although it is not required, the RStudio IDE can make a lot of things
much easier when you work on R-related projects. The RStudio IDE can be
downloaded from \url{https://www.rstudio.com}.

\section{Pandoc}\label{pandoc}

An R Markdown document (\texttt{*.Rmd}) is first compiled to Markdown
(\texttt{*.md}) through the \textbf{knitr} package, and then Markdown is
compiled to other output formats (such as LaTeX or HTML) through
Pandoc.\index{Pandoc} This process is automated by the
\textbf{rmarkdown} package. You do not need to install \textbf{knitr} or
\textbf{rmarkdown} separately, because they are the required packages of
\textbf{bookdown} and will be automatically installed when you install
\textbf{bookdown}. However, Pandoc is not an R package, so it will not
be automatically installed when you install \textbf{bookdown}. You can
follow the installation instructions on the Pandoc homepage
(\url{http://pandoc.org}) to install Pandoc, but if you use the RStudio
IDE, you do not really need to install Pandoc separately, because
RStudio includes a copy of Pandoc. The Pandoc version number can be
obtained via:

\begin{Shaded}
\begin{Highlighting}[]
\NormalTok{rmarkdown}\OperatorTok{::}\KeywordTok{pandoc_version}\NormalTok{()}
\NormalTok{## [1] '1.19.2.1'}
\end{Highlighting}
\end{Shaded}

If you find this version too low and there are Pandoc features only in a
later version, you can install the later version of Pandoc, and
\textbf{rmarkdown} will call the newer version instead of its built-in
version.

\section{LaTeX}\label{latex}

LaTeX\index{LaTeX} is required only if you want to convert your book to
PDF. The typical choice of the LaTeX distribution depends on your
operating system. Windows users may consider MiKTeX
(\url{http://miktex.org}), Mac OS X users can install MacTeX
(\url{http://www.tug.org/mactex/}), and Linux users can install TeXLive
(\url{http://www.tug.org/texlive}). See
\url{https://www.latex-project.org/get/} for more information about
LaTeX and its installation.

Most LaTeX distributions provide a minimal/basic package and a full
package. You can install the basic package if you have limited disk
space and know how to install LaTeX packages later. The full package is
often significantly larger in size, since it contains all LaTeX
packages, and you are unlikely to run into the problem of missing
packages in LaTeX.

LaTeX error messages may be obscure to beginners, but you may find
solutions by searching for the error message online (you have good
chances of ending up on
\href{http://tex.stackexchange.com}{StackExchange}). In fact, the LaTeX
code converted from R Markdown should be safe enough and you should not
frequently run into LaTeX problems unless you introduced raw LaTeX
content in your Rmd documents. The most common LaTeX problem should be
missing LaTeX packages, and the error may look like this:

\begin{Shaded}
\begin{Highlighting}[]
\NormalTok{! LaTeX Error: File `titling.sty' not found.}

\NormalTok{Type X to quit or <RETURN> to proceed,}
\NormalTok{or enter new name. (Default extension: sty)}

\NormalTok{Enter file name: }
\NormalTok{! Emergency stop.}
\NormalTok{<read *> }
         
\NormalTok{l.107 ^^M}

\NormalTok{pandoc: Error producing PDF}
\NormalTok{Error: pandoc document conversion failed with error 43}
\NormalTok{Execution halted}
\end{Highlighting}
\end{Shaded}

This means you used a package that contains \texttt{titling.sty}, but it
was not installed. LaTeX package names are often the same as the
\texttt{*.sty} filenames, so in this case, you can try to install the
\texttt{titling} package. Both MiKTeX and MacTeX provide a graphical
user interface to manage packages. You can find the MiKTeX package
manager from the start menu, and MacTeX's package manager from the
application ``TeX Live Utility''. Type the name of the package, or the
filename to search for the package and install it. TeXLive may be a
little trickier: if you use the pre-built TeXLive packages of your Linux
distribution, you need to search in the package repository and your
keywords may match other non-LaTeX packages. Personally, I find it
frustrating to use the pre-built collections of packages on Linux, and
much easier to install TeXLive from source, in which case you can manage
packages using the \texttt{tlmgr} command. For example, you can search
for \texttt{titling.sty} from the TeXLive package repository:

\begin{Shaded}
\begin{Highlighting}[]
\ExtensionTok{tlmgr}\NormalTok{ search --global --file titling.sty}
\CommentTok{# titling:}
\CommentTok{#    texmf-dist/tex/latex/titling/titling.sty}
\end{Highlighting}
\end{Shaded}

Once you have figured out the package name, you can install it by:

\begin{Shaded}
\begin{Highlighting}[]
\ExtensionTok{tlmgr}\NormalTok{ install titling  # may require sudo}
\end{Highlighting}
\end{Shaded}

LaTeX distributions and packages are also updated from time to time, and
you may consider updating them especially when you run into LaTeX
problems. You can find out the version of your LaTeX distribution by:

\begin{Shaded}
\begin{Highlighting}[]
\KeywordTok{system}\NormalTok{(}\StringTok{'pdflatex --version'}\NormalTok{)}
\NormalTok{## pdfTeX 3.14159265-2.6-1.40.18 (TeX Live 2017)}
\NormalTok{## kpathsea version 6.2.3}
\NormalTok{## Copyright 2017 Han The Thanh (pdfTeX) et al.}
\NormalTok{## There is NO warranty.  Redistribution of this software is}
\NormalTok{## covered by the terms of both the pdfTeX copyright and}
\NormalTok{## the Lesser GNU General Public License.}
\NormalTok{## For more information about these matters, see the file}
\NormalTok{## named COPYING and the pdfTeX source.}
\NormalTok{## Primary author of pdfTeX: Han The Thanh (pdfTeX) et al.}
\NormalTok{## Compiled with libpng 1.6.29; using libpng 1.6.29}
\NormalTok{## Compiled with zlib 1.2.11; using zlib 1.2.11}
\NormalTok{## Compiled with xpdf version 3.04}
\end{Highlighting}
\end{Shaded}

\bibliography{book.bib,packages.bib}

\backmatter
\printindex

\end{document}
