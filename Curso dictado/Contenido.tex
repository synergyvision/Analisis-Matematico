\documentclass[12pt,letterpaper]{article}
\usepackage[OT1]{fontenc}
\usepackage[utf8]{inputenc}
\usepackage{graphicx}
\usepackage{xspace}
\def\courssigle{\includegraphics[scale=0.04]{Vision.png} \xspace}
\def\courstitre{Análisis moderno}
\def\session{Semestre 01-2018}

%%%%%%%%%%%%%%%%%%%%%%%%%%%%%%%%%%%%%%%%%%%%%%%%%%%%%%%%%%%%%%%%%%%%%%%%%%%
%% The usual suspects
%%%%%%%%%%%%%%%%%%%%%%%%%%%%%%%%%%%%%%%%%%%%%%%%%%%%%%%%%%%%%%%%%%%%%%%%%%%
\usepackage{amsmath}
\usepackage{amssymb}
\usepackage{hyperref}
%\usepackage{times}
%\usepackage{enumitem}
\usepackage[final]{pdfpages}
\usepackage{booktabs}
\usepackage{multirow}
\usepackage{longtable}
\usepackage{array}


%%%%%%%%%%%%%%%%%%%%%%%%%%%%%%%%%%%%%%%%%%%%%%%%%%%%%%%%%%%%%%%%%%%%%%%%%%%
%% French
%%%%%%%%%%%%%%%%%%%%%%%%%%%%%%%%%%%%%%%%%%%%%%%%%%%%%%%%%%%%%%%%%%%%%%%%%%%

%\usepackage[canadien]{babel}
%\frenchspacing

%%%%%%%%%%%%%%%%%%%%%%%%%%%%%%%%%%%%%%%%%%%%%%%%%%%%%%%%%%%%%%%%%%%%%%%%%%%
%% Date, Location, Contact Info
%%%%%%%%%%%%%%%%%%%%%%%%%%%%%%%%%%%%%%%%%%%%%%%%%%%%%%%%%%%%%%%%%%%%%%%%%%%

\def\Date{\null\hfill\makebox{\small\today}}
\def\Location{\null\hfill\makebox{Caracas, Venezuela}}

%%%%%%%%%%%%%%%%%%%%%%%%%%%%%%%%%%%%%%%%%%%%%%%%%%%%%%%%%%%%%%%%%%%%%%%%%%%
%% Margins
%%%%%%%%%%%%%%%%%%%%%%%%%%%%%%%%%%%%%%%%%%%%%%%%%%%%%%%%%%%%%%%%%%%%%%%%%%%

\usepackage[paper=letterpaper,margin=0.75in]{geometry}

%%%%%%%%%%%%%%%%%%%%%%%%%%%%%%%%%%%%%%%%%%%%%%%%%%%%%%%%%%%%%%%%%%%%%%%%%%%
%% Header, Footer, Page Numbering
%%%%%%%%%%%%%%%%%%%%%%%%%%%%%%%%%%%%%%%%%%%%%%%%%%%%%%%%%%%%%%%%%%%%%%%%%%%
\usepackage{fancyhdr}
\usepackage{lastpage}
%\setcounter{page}{12}

\setlength{\headwidth}{\textwidth}
\pagestyle{fancy}
  \lhead{\sc \courssigle}
  \chead{\sc \courstitre}
  \rhead{\sc \session}
  %\lfoot{Département de Mathématiques, Université du Québec à Montréal}
  \lfoot{}
  \cfoot{}
  \rfoot{Página~\thepage~de~\pageref{LastPage}}
\renewcommand{\headrulewidth}{0.4pt}
%\renewcommand{\footrulewidth}{0.4pt}

%%%%%%%%%%%%%%%%%%%%%%%%%%%%%%%%%%%%%%%%%%%%%%%%%%%%%%%%%%%%%%%%%%%%%%%%%%%
%% set default spacing
%%%%%%%%%%%%%%%%%%%%%%%%%%%%%%%%%%%%%%%%%%%%%%%%%%%%%%%%%%%%%%%%%%%%%%%%%%%
%\setlength{\parindent}{0in}
\renewcommand{\baselinestretch}{1.0}

%%%%%%%%%%%%%%%%%%%%%%%%%%%%%%%%%%%%%%%%%%%%%%%%%%%%%%%%%%%%%%%%%%%%%%%%%%%
%% Notes, Todos, Tasks, etc.
%%%%%%%%%%%%%%%%%%%%%%%%%%%%%%%%%%%%%%%%%%%%%%%%%%%%%%%%%%%%%%%%%%%%%%%%%%%
\usepackage{boxedminipage}
\newenvironment{note}[1][Note]
   {\bigskip\begin{center}\begin{boxedminipage}{4.5in}\setlength{\parindent}{1em}\noindent\textbf{#1. }}
   {\end{boxedminipage}\end{center}\bigskip}
\newenvironment{todo}{\begin{note}[Todo]}{\end{note}}

%%%%%%%%%%%%%%%%%%%%%%%%%%%%%%%%%%%%%%%%%%%%%%%%%%%%%%%%%%%%%%%%%%%%%%%%%%%
%% Some macros
%%%%%%%%%%%%%%%%%%%%%%%%%%%%%%%%%%%%%%%%%%%%%%%%%%%%%%%%%%%%%%%%%%%%%%%%%%%
\def\emphasize#1{\emph{\textbf{#1}}}


\newtheorem{ex}{\bf Ejercicio}
\renewcommand{\thefootnote}{\fnsymbol{footnote}}

\newcommand{\V}{{\bf{V}}}
\newcommand{\F}{{\bf{F}}}
\newcommand\tab[1][1.5cm]{\hspace*{#1}}

%%%%%%%%%%%%%%%%%%%%%%%%%%%%%%%%%%%%%%%%%%%%%%%%%%%%%%%%%%%%%%%%%%%%%%%%%%%
% Document
%%%%%%%%%%%%%%%%%%%%%%%%%%%%%%%%%%%%%%%%%%%%%%%%%%%%%%%%%%%%%%%%%%%%%%%%%%%

\begin{document}

\begin{center}
\Large\sc
Plan del curso\\
\end{center}

\bigskip

\textsc{Profesor}: \\
\tab {\sc Yannic VARGAS.}\\
\tab Correo: \texttt{yv@synergy.vision}

\

\textsc{Horario (tentativo)}: \\
\tab lunes, 8h a 9h30, Honey Comb. \\
\tab jueves, 8h a 9h30, Honey Comb.

\

\textsc{Objetivo del curso}: \\
\tab Hacer un estudio detallado de conceptos y resultados clásicos del análisis matemático, \\
\tab las probabilidades y la teoría de la medida, con énfasis en la aplicación a las finanzas.

\

\textsc{Material previo}: \\
\tab El curso está estructurado de manera que todos los temas se aborden en profundidad,\\
\tab a partir de nociones básicas vistas en clase. Sin embargo, muchos temas que anteceden \\
\tab a este curso no serán discutidos por motivos de tiempo. En las referencias siguientes se \\
\tab pueden encontrar material para completar los tópicos que anteceden a este curso:

\medskip

\tab - Bloch, Ethan, Proofs and Fundamentals, Birkha\"{u}ser, 2000.\\
\tab \  \quad - Maia, Manuel, Lógica proposicional, teoremas y demostraciones, 2012. \\
\tab \qquad {\footnotesize \url{https://upload.wikimedia.org/wikipedia/commons/b/b4/Logica_y_demostraciones.pdf}}\\
\tab \quad \ - Sibley, Thomas, Foundations of Mathematics, John Wiley \& Sons 2009.\\

\

\textsc{Contenido}: \\

\tab \textsc{Parte I: Análisis real}\\
\tab \qquad \textbf{Conjuntos y funciones:} operaciones, conjuntos numéricos, relaciones,\\
	\tab \qquad \qquad funciones, cardinalidad.\\
	\tab \qquad \textbf{Sistema numérico real y complejo}: propiedades algebraicas, estructura \\
	\tab \qquad \qquad de orden, completitud, inducción matemática, espacios Euclídeos.\\
\tab \qquad \textbf{Estructuras algebraicas:} semigrupos y grupos, espacios vectoriales, \\
	\tab \qquad \qquad transformaciones lineales, espacios vectoriales y cocientes, álgebras. \\
\tab \qquad \textbf{Sucesiones numéricas:} límite de una sucesión, sucesiones monótonas, \\
	\tab \qquad \qquad subsucesiones, límite inferior y superior.\\
\tab \qquad \textbf{Sucesiones y series:} límite de una función, límite inferior y superior, \\
	\tab \qquad \qquad funciones continuas, propiedades, continuidad uniforme.\\
\tab \qquad \textbf{Diferenciación:} definición y ejemplos, el Teorema del Valor Medio, \\
	\tab \qquad \qquad funciones convexas, funciones inversas, Regla de L'Hospital, Teorema \\ 
	\tab \qquad \qquad de Taylor en $\mathbb{R}$, método de Newton.\\
\tab \qquad \textbf{Integración de Riemann:} integral de Riemann-Darboux, propiedades de \\
	\tab \qquad \qquad la integral, evaluación de la integral, fórmula de Stirling,\\ 
	\tab \qquad \qquad versión integral del Teorema del Valor Medio, estimación de la integral, \\
	\tab \qquad \qquad integrales impropias, la integrabilidad según Riemann, funciones a \\ 
	\tab \qquad \qquad variación acotada, la integral de Riemann-Stieltjes.\\
\tab \qquad \textbf{Series numéricas infinitas:} definición y ejemplos, series con términos \\
	\tab \qquad \qquad no-negativos, criterios de convergencia, convergencia condicional y \\
	\tab \qquad \qquad absoluta, sucesiones dobles y series. \\
\tab \qquad \textbf{Sucesiones y series de funciones:} convergencia de sucesiones de funciones, \\
	\tab \qquad \qquad propiedades del límite de funciones, convergencia de las series de funciones, \\
	\tab \qquad \qquad series de potencia.\\
\tab \qquad \textbf{Funciones en varias variables:} transformaciones lineales, diferenciación, \\
	\tab \qquad \qquad Principio de la Contracción, Teorema de la Función Inversa, Teorema de \\
	\tab \qquad \qquad la Función Implícita, Teorema del Rango, determinantes, derivadas de \\
	\tab \qquad \qquad orden superior, diferenciación de integrales.\\
\tab \qquad \textbf{Integración de formas diferenciales:} integración, aplicaciones de primitivas, \\	
	\tab \qquad \qquad cambio de variables, formas diferenciales, cadenas y símplices, Teorema de \\
	\tab \qquad \qquad Stoke, formas cerradas y formas exactas, análisis vectorial.\\

\tab \textsc{Parte II: Principios de topología}\\
\tab \qquad \textbf{Espacios métricos.}  \\
\tab \qquad \textbf{Espacios lineales y normados:} normas y seminormas, completación de un \\
	\tab \qquad \qquad espacio normado, series infinitas en espacios normados, sumas \\
	\tab \qquad \qquad no-ordenadas en espacios normados, transformaciones lineales acotadas, \\
	\tab \qquad \qquad álgebras de Banach. \\
\tab \qquad \textbf{Espacios topológicos:} abiertos y cerrados, sistemas de entornos, bases de \\
	\tab \qquad \qquad entornos, topología relativa, nets.\\
\tab \qquad \textbf{Continuidad en espacios topológicos:} propiedades generales, topologías \\
	\tab \qquad \qquad iniciales, topología producto, topología cociente, espacio de funciones \\
	\tab \qquad \qquad continuas, conjuntos F-sigma y G-delta.\\
\tab \qquad \textbf{Espacios topológicos normados:} Lema de Urysohn, Teorema de \\
	\tab \qquad \qquad Extensión de Tietze.\\
\tab \qquad \textbf{Espacios topológicos compactos:} convergencia en espacios compactos, \\ 
	\tab \qquad \qquad compacidad del producto cartesiano, continuidad y compacidad.\\
\tab \qquad \textbf{Espacios métricos totalmente acotados.}\\
\tab \qquad \textbf{Equicontinuidad.}\\
\tab \qquad \textbf{Teorema de Stone-Weierstrass.}\\
\tab \qquad \textbf{Espacios topológicos localmente compactos:} propiedades generales, \\
	\tab \qquad \qquad funciones a soporte compacto, funciones que se anulan en al infinito, \\
	\tab \qquad \qquad compactificación a un punto.\\
\tab \qquad \textbf{Espacio de funciones diferenciables.}\\
\tab \qquad \textbf{Particiones de la unidad.}\\
\tab \qquad \textbf{Conexidad.}\\
\tab \qquad \textbf{Espacios convexos:} familias de seminormas, Teorema de Separación y \\
	\tab \qquad \qquad de Prolongamiento, Teorema de Krein-Milman.\\

\tab \textsc{Parte III: Medida e integración}\\
\tab \qquad \textbf{Conjuntos medibles:} introducción, espacios medibles, medidas, espacios \\
	\tab \qquad \qquad medibles completos, medida externa y medibilidad, extensión de una \\
	\tab \qquad \qquad medida, medida de Lebesgue, medida de Lebesgue-Stieltjes, \\
	\tab \qquad \qquad conjuntos especiales.\\
\tab \qquad \textbf{Funciones medibles:} transformaciones medibles, funciones numéricas \\
	\tab \qquad \qquad medibles, funciones simples, convergencia de funciones medibles.\\
\tab \qquad \textbf{Integración:} construcción de la integral, propiedades básicas, conexiones \\
	\tab \qquad \qquad con la integral de Riemann en $\mathbb{R}^n$, teoremas de convergencia, \\
	\tab \qquad \qquad integración sobre una medida producto, aplicaciones del Teorema \\
	\tab \qquad \qquad de Fubini.\\
\tab \qquad \textbf{Espacios $L^p$:} definición y propiedades generales, aproximación en $L^p$, \\
	\tab \qquad \qquad convergencia en $L^p$, integrabilidad uniforme, funciones convexas y \\
	\tab \qquad \qquad desigualdad de Jensen.\\
\tab \qquad \textbf{Diferenciación:} medidas con signo, medidas complejas, continuidad \\
	\tab \qquad \qquad absoluta de medidas, diferenciación de medidas, funciones a \\
	\tab \qquad \qquad variación acotada, funciones absolutamente continuas.\\
\tab \qquad \textbf{Análisis de Fourier en $\mathbb{R}^n$:} convolución de funciones, la transformada \\
	\tab \qquad \qquad de Fourier, funciones de rápido decrecimiento, análisis de Fourier \\
	\tab \qquad \qquad de medidas en $\mathbb{R}^n$.\\ 
\tab \qquad \textbf{Medidas en espacios localmente compactos:} medidas de Radon, \\
	\tab \qquad \qquad Teorema de Representación de Riesz, productos de medidas de Radon, \\
	\tab \qquad \qquad convergencia ``vague", Teorema de Representación de Daniell-Stone.\\
	
\tab \textsc{Parte IV: Análisis funcional}\\
\tab \qquad \textbf{Espacios de Banach:} espacios normados, separación de conjuntos convexos, \\
 	\tab \qquad \qquad Teorema del Prolongamiento, duales de los espacios $\ell^p$, convergencia débil, \\
	\tab \qquad \qquad Teorema de Banach-Steinhaus, espacios reflexivos, operadores contínuos y \\
	\tab \qquad \qquad compactos, Teorema de Fredholm-Riesz, aplicaciones abiertas y grafos \\
	\tab \qquad \qquad cerrados, caso complejo.\\
\tab \qquad \textbf{Espacios localmente convexos:} propiedades generales, funcionales lineales \\
	\tab \qquad \qquad contínuos, Teoremas de Separación de Hahn-Banach, algunas \\
	\tab \qquad \qquad construcciones.\\
\tab \qquad \textbf{Topologías débiles en espacios normados:} topología débil, espacios \\
	\tab \qquad \qquad reflexivos, espacios uniformemente convexos.\\
\tab \qquad \textbf{Espacios de Hilbert:} principios generales, ortogonalidad, bases ortonormales, \\
	\tab \qquad \qquad el adjunto del espacio de Hilbert.\\
\tab \qquad \textbf{Teoría de operadores:} tipos de operadores, operadores compactos y \\
	\tab \qquad \qquad de rango finito, el Teorema Espectral para operadores normales \\
	\tab \qquad \qquad compactos, el álgebra del grupo $L^1$, representaciones, grupos abelianos \\
	\tab \qquad \qquad localmente compactos.\\
\tab \qquad \textbf{Álgebras de Banach:} principios básicos, Teoría Espectral, el espectro de un \\
	\tab \qquad \qquad álgebra, Teoría de Gelfand, el caso no-unitario, cálculo de operadores.\\ 
	
\tab \textsc{Parte V: Aplicaciones}\\
\tab \qquad \textbf{Distribuciones:} teoría general, operaciones en distribuciones, distribuciones \\
	\tab \qquad \qquad a soportes compactos, convolución de distribuciones, Teoría de Sobolev.\\
\tab \qquad \textbf{Análisis en grupos localmente compactos:} grupos topológicos, medida \\
	\tab \qquad \qquad de Haar, representaciones, grupos abelianos localmente compactos.\\
\tab \qquad \textbf{Análisis en semigrupos:} semigrupos con topologías, funciones débilmente \\
	\tab \qquad \qquad quasi-periódicas, la estructura de los semigrupos compactos, funciones \\
	\tab \qquad \qquad fuertemente quasi-periódicas, semigrupo de operadores.\\
\tab \qquad \textbf{Teoría de probabilidades:} variables aleatorias, independencia, esperanza \\
	\tab \qquad \qquad condicional, sucesiones de variables aleatorias independientes, martingalas \\
	\tab \qquad \qquad a tiempo discreto, procesos estocásticos, movimiento browniano. \\
\tab \qquad \textbf{Integración estocástica:} integral de Ito para procesos simples, integral de \\ 
	\tab \qquad \qquad Ito generalizada, integral de Ito como una martingala.\\
\tab \qquad \textbf{Aplicación a las finanzas:} el proceso de precios de Stock, portafolios \\
	\tab \qquad \qquad autofinanciados, opciones de llamadas, opción de precios de \\
	\tab \qquad \qquad Black-Scholes.\\

\

\textsc{Referencias}: \\
\tab Apostol, Thom, Mathematical Analysis, 2nd ed. Addison-Wesley, 2000\\
\tab R. Ash and C. Doleans-Dade, Probability and Measure Theory, 2nd Ed., \\
\tab \qquad \qquad Academic Press, San Diego, 2000.\\
\tab H. Brezis, Functional Analysis, Sobolev Spaces, and Partial Differential Equations, \\
\tab \qquad \qquad Springer-Verlag, New York, 2011.\\
\tab J. Conway, A Course in Functional Analysis, Springer-Verlag, New York, 1990.\\
\tab Dudley, Richard, Real Analysis and Probability, Cambridge University
Press, 2002.\\
\tab G. Folland, Real Analysis. Modern Techniques and Their Applications, \\
\tab \qquad \qquad 2nd Ed. John Wiley \& Sons, New York, 1999.\\
\tab H. Junghenn, Option Valuation: A First Course in Financial Mathemtics, CRC Press,\\
\tab \qquad \qquad Boca Raton, 2012.\\
\tab H. Junghenn, A Course in Real Analysis, CRC Press, Boca Raton, 2015.\\
\tab S. Lang, Real and Functional Analysis, 3rd Ed., Springer-Verlag, New York, 1993.\\
\tab I. Rana, An Introduction to Measure and Integration, 2nd Ed., Graduate Studies \\
\tab \qquad \qquad in Mathematics Vol. 45, AMS, Providence, 2002.\\
\tab Rudin, Walter, Principles of Mathematical Analysis, 3rd ed. McGraw-Hill,
1976.\\
\tab S. Shreve, Stochastic Calculus for Finance, Springer-Verlag, New York, 2004.\\
\tab Strang, Gilbert, Linear Algebra and Its Application, 4th ed. Thomson \\
\tab \qquad \qquad  Brooks/Cole, 2006.\\

\

\textsc{Evaluación}: \\
\tab La evaluación consistirá de una serie de trabajos y exposiciones orales a lo largo \\
\tab del semestre.









\end{document}